\documentclass[twocolumn]{aastex631}
\bibliographystyle{aasjournal}

\usepackage{graphicx}
\usepackage[caption=false]{subfig}
\usepackage{amsmath}
\usepackage{booktabs}
\usepackage{censor}

\let\pwiflocal=\iffalse \let\pwifjournal=\iffalse
%From: http://arxiv.org/format/1512.00483
%\input{setup}
%\input{mgs_setup}

%% General
\newcommand{\msini}{\ensuremath{m \sin i}}
\newcommand{\mplsini}{\ensuremath{\mpl\sin i}}
\newcommand{\teff}{\ensuremath{T_{\rm eff}}}
\newcommand{\logg}{\ensuremath{\log{g}}}
\newcommand{\vsini}{\ensuremath{v \sin{I_\star}}}
\newcommand{\feh}{\ensuremath{\rm [Fe/H]}}
\newcommand{\logl}{\ensuremath{\log{L}}}
\newcommand{\vmac}{\ensuremath{v_{\rm mac}}}
\newcommand{\vmic}{\ensuremath{v_{\rm mic}}}
% Activity index R'_HK
\newcommand{\rhk}{\ensuremath{R^{\prime}_{HK}}}
% log of R'_HK
\newcommand{\logrhk}{\ensuremath{\log\rhk}}
% S average value
\newcommand{\Savg}{\ensuremath{\langle S\rangle}}


%% ---------------------------------------------------------------------
%% Solar quantities 
\newcommand{\rsun}{\ensuremath{R_\sun}}
\newcommand{\msun}{\ensuremath{M_\sun}}
\newcommand{\lsun}{\ensuremath{L_\sun}}
\newcommand{\loglsun}{\ensuremath{\log{L_\sun}}}
\newcommand{\teffsun}{\ensuremath{T_{eff,\sun}}}
\newcommand{\rhosun}{\ensuremath{\rho_\sun}}
\newcommand{\loggsun}{\ensuremath{\log{g_{\sun}}}}

%% ---------------------------------------------------------------------
%% Stellar quantities 
\newcommand{\rstar}{\ensuremath{R_\star}}
\newcommand{\mstar}{\ensuremath{M_\star}}
\newcommand{\lstar}{\ensuremath{L_\star}}
\newcommand{\astar}{\ensuremath{a_\star}}
\newcommand{\loglstar}{\ensuremath{\log{L_\star}}}
\newcommand{\teffstar}{\ensuremath{T_{\rm eff\star}}}
\newcommand{\rhostar}{\ensuremath{\rho_\star}}
\newcommand{\loggstar}{\ensuremath{\log{g_{\star}}}}

%% ---------------------------------------------------------------------
%% Earth
\newcommand{\rearth}{\ensuremath{R_\earth}}
\newcommand{\mearth}{\ensuremath{M_\earth}}
\newcommand{\learth}{\ensuremath{L_\earth}}
\newcommand{\teffearth}{\ensuremath{T_{\rm eff,\earth}}}
\newcommand{\rhoearth}{\ensuremath{\rho_\earth}}

%% ---------------------------------------------------------------------
%% Planetary
\newcommand{\rpl}{\ensuremath{R_{p}}}
\newcommand{\mpl}{\ensuremath{M_{p}}}
\newcommand{\lpl}{\ensuremath{L_{p}}}
\newcommand{\teffpl}{\ensuremath{T_{\rm eff,{p}}}}
\newcommand{\rhopl}{\ensuremath{\rho_{p}}}
\newcommand{\ipl}{\ensuremath{i_{p}}}
\newcommand{\epl}{\ensuremath{e_{p}}}
\newcommand{\gpl}{\ensuremath{g_{p}}}
\newcommand{\loggpl}{\ensuremath{\log g_{p}}}

\newcommand{\arstar}{\ensuremath{a/\rstar}}
\newcommand{\zrstar}{\ensuremath{\zeta/\rstar}}

%% ---------------------------------------------------------------------
%% Jupiter
\newcommand{\rjup}{\ensuremath{R_{\rm J}}}
\newcommand{\mjup}{\ensuremath{M_{\rm J}}}
\newcommand{\ljup}{\ensuremath{L_{\rm J}}}
\newcommand{\teffjup}{\ensuremath{T_{eff,{\rm J}}}}
\newcommand{\rhojup}{\ensuremath{\rho_{\rm J}}}
\newcommand{\gjup}{\ensuremath{\g_{\rm J}}}

\newcommand{\rjuplong}{\ensuremath{R_{\rm Jup}}}
\newcommand{\mjuplong}{\ensuremath{M_{\rm Jup}}}
\newcommand{\ljuplong}{\ensuremath{L_{\rm Jup}}}
\newcommand{\teffjuplong}{\ensuremath{T_{eff,{\rm Jup}}}}
\newcommand{\rhojuplong}{\ensuremath{\rho_{\rm Jup}}}
\newcommand{\gjuplong}{\ensuremath{\g_{\rm Jup}}}

\providecommand{\eprint}[1]{\href{http://arxiv.org/abs/#1}{#1}}
\providecommand{\adsurl}[1]{\href{#1}{ADS}}
\newcommand{\project}[1]{\textsl{#1}}

\begin{document}
\shorttitle{HPF spectral analysis}
\shortauthors{TBD}

\title{Atmospheric escape and star-planet-interaction in the HAT-P-67 b system}

\author{TBD}
\affiliation{TBD}


\begin{abstract}
    We aim to understand how exoplanet atmospheres evolve across age, stellar irradiation environments, planet properties, and---for extremely close-in exoplanets---direct star-planet interaction (SPI) such as stellar winds, tidal forces, and magnetic interactions. The scarcity of exoplanet atmosphere measurements across a range of stellar properties has limited our ability to complete this picture.

    The Helium Exospheres program with the Habitable Zone Planet Finder (HPF) spectrograph on the Hobby Eberly Telescope (HET) in West Texas has targeted 23 exoplanet systems over two years. Here, we present a study of HAT-P-67, an F subgiant hosting a low density planet with 0.3 $M_{\mathrm{Jup}}$ and 2.1 $R_{\mathrm{Jup}}$ on a close-in 4.8 day orbit. The high energy radiation environment should provide a uniquely harsh testing ground for examining atmospheric escape. The planet shows prominent in-transit absorption in the He 10830 \AA~ triplet and infrared Ca II 8542 \AA~ triplet across 20 visits over the course of a year including in- and out-of-transit spectroscopy. We identify the rotational period of HAT-P-67 to be 5.417 days by using contemporaneous data from the Transiting Exoplanet Survey Satellite (TESS). We consider multiple interpretations of the HPF and TESS observations. First, the period measured by TESS may be the rotational period of the star and thus, the star and planet are nearly-tidally locked. On the other hand, the period measured by TESS may result from SPI-induced hot spots on the surface that follow the magnetic field lines linked to the planet's orbital period. Finally, we consider the possibility of stellar surface inhomogeneity contaminating the He and Ca II transit signals.
\end{abstract}

\keywords{stars: fundamental parameters ---  stars: statistics}

\section{Introduction}\label{sec:intro}

Exoplanets possess a dynamic range of atmospheric properties far beyond what we have seen in our own Solar System.  Extreme radiation effects, atmospheric erosion, and unexpected chemistry must exist, but collecting examples of these phenomena remains at the edge of the possible today. The minuscule signal strengths of atmospheric shells has hidden all but the most conspicuous phenomena.  Helium 10830 \AA has emerged as the most conspicuous exosphere indicator, enabling ground-based detection with high-resolution near-infrared (IR) spectrographs.

At least 8 systems have Helium 10830 exosphere measurements to date: \object[GJ 3470]{GJ3470} \citep{2020ApJ...894...97N, 2021A&A...647A.129L}, \object{HD 63433 c} \citep{2022AJ....163...68Z}, WASP 107 b \citep{2019A&A...623A..58A,2020AJ....159..115K}, HD 209458 b \citep{2019A&A...629A.110A}, WASP 69 b \citep{2020AJ....159..278V}, WASP 52 b \citep{2020AJ....159..278V}, HD 189733 b \citep{2021A&A...647A.129L}, and HAT-P-18 b \citep{2021ApJ...909L..10P}.


Here we present a study of the exosphere of HAT-P-67b.  HAT-P-67 is an F type sub-giant with the Saturn-sized planet HAT-P-67b.  HAT-P-67b stands out for a few reasons.  It has a relatively low surface gravity, which could make it amenable to rapid atmospheric mass loss.  It orbits an evolved star, which means its instellation may have increased as the star's size has grown.  The star appears close to the star, making it conceivable that star-planet interactions could be at play.  These and other reasons make HAT-P-67b a unique laboratory for measuring the properties of exospheres and their interactions with their host stars.  The system appears to consist of a binary with an M dwarf companion HAT-P-67B separated by $9\farcs0$.

\section{Observations}
\subsection{Habitable Zone Planet Finder (HPF)}

The Habitable Zone Planet Finder Spectrograph \citep[HPF][]{2012SPIE.8446E..1SM,2014SPIE.9147E..1GM, 2019Optic...6..233M} on the queue-scheduled 10-meter \emph{Hobby-Eberly Telescope} \citep[HET][]{1998SPIE.3352...34R, 2007PASP..119..556S} operates in the near-IR from $8100-12800~\AA\;$ spanning the \textit{z}, \textit{Y}, and \textit{J} bands. We observed HAT-P-67b with HPF in-transit on 2020 April 28, 2020 May 22, and 2020 June 15 and out-of-transit on a total of \censorbox{xx} other nights.


\begin{deluxetable*}{lccc}[!ht]
    \tablecaption{HPF Observations for HAT-P-67b \label{tab:observation_properties}}
    \tablehead{
        \colhead{Date} & \colhead{Time of Observation} & \colhead{Transit}}
    \startdata
    2020-04-28 & 06:10:31 & In transit\\
    2020-04-28 & 06:16:12 & In transit\\
    2020-04-28 & 06:21:53 & In transit\\
    2020-04-28 & 06:27:34 & In transit\\
    2020-04-28 & 06:33:15 & In transit\\
    2020-04-28 & 06:38:57 & In transit\\
    2020-04-28 & 06:44:38 & In transit\\
    2020-04-28 & 06:50:19 & In transit\\
    2020-04-28 & 06:10:31 & In transit\\
    2020-04-28 & 06:10:31 & In transit\\
    2020-04-28 & 06:10:31 & In transit\\
    2020-04-28 & 06:10:31 & In transit\\
    2020-04-28 & 06:10:31 & In transit
    \enddata
\end{deluxetable*}

\subsection{TESS Light Curve}
HAT-P-67 was observed with the \emph{Transiting Exoplanet Survey Satellite} \citep[TESS][]{2014SPIE.9143E..20R} in Sectors 24, 26, 51, 52, 53 with 2-minute cadence, and in Sector 25 with 30-minute (FFI) cadence.  The Sector 25 FFI data appeared malformed, possibly due to problems with pipeline CCD smear correction, and was disregarded.

\subsection{ASAS-SN}
We retrieved ground-based photometry with the \emph{All-Sky Automated Survey for Supernovae} (ASAS-SN) using the Sky-Portal \citep{shappee14,2017PASP..129j4502K}.

\subsection{DASCH}
We searched for observations of HAT-P-67 in other archives.  It appears in the Harvard Plate Archive, with 5809 measurements digitized through DASCH \censorbox{citation}, spanning over 120 years of coarse photometric monitoring.  In principle these datasets could inform long-term variability trends such as stellar cycles.  In practice the 0.15 magnitude jitter appears too coarse to perceive any genuine astrophysical variability, with no conspicuous trend seen.  We can therefore place a relatively mild constraint that the star is stable at the $30\%$ level over periods of tens of years.

\subsection{Gaia DR3}
HAT-P-67A (\emph{Gaia DR3 1358614983131339392}) has a parallax of $2.69\pm0.01$ mas in \emph{Gaia} DR3, placing it at about 372$\pm$1.4 pc, slightly farther than previously estimated by \citet{2017AJ....153..211Z} using Gaia DR1. Its coarse Gaia-based estimates for RV and Doppler broadening are consistent with the more precise values published in \citet{2017AJ....153..211Z}.  Its wide companion HAT-P-67B (\emph{Gaia DR3 1358614983131339904}) \citep{2019MNRAS.490.5088M} has statistically identical parallax ($2.58\pm0.05$) and proper motions, confirming its interpretation as co-moving, with projected separation of about 3400 AU.

\subsection{Other archives}
We searched for observations of HAT-P-67 in public archives.  It has one $R\sim8000$ spectrum from the Intermediate Dispersion Spectrograph on the INT, centered near 4000 $\AA$.  It has 26 photometry points from \emph{Pan-STARRS} spread across its 5 filters.  The other archival observations have been previously reported by \citet{2017AJ....153..211Z}, including Keck HIRES, Keck NIRC2, and WIYN High-Resolution Infrared Camera.  The northern source does not appear in ESO archives, nor the Gemini Archive, and it is not publicly accessible in the Zwicky Transient Factory (ZTF) archive.  It was not targeted by Spitzer or Hubble.



\section{Analysis}

\begin{figure}
    \includegraphics[width=\linewidth]{../../figures/Gaia_HAT_P_67_AB.png}
    \caption{HR Diagram of F star HAT-P-67A with its assumed M-dwarf main sequence companion HAT-P-67B.  The density of stars comes from the \emph{Kepler} prime field estimated by \citet{2020AJ....160..108B} using \emph{Gaia DR2}.}
    \label{fig:gaiaHRD}
\end{figure}



Ideally we seek the pixel-to-pixel substructured spectrum of the helium exosphere, corrected for all the conceivable instrumental, telluric, and stellar contamination artifacts.  In practice, these corrections are incomplete and so some care is needed to assess the extent of contamination.  We experimented with a range of pre-processing and analysis techniques to quantify the accuracy and precision of our HPF spectra.


\subsection{TESS Light Curve}

\subsubsection{Revised exoplanet parameters with exoplanet framework}
We assembled a composite lightcurve by stitching TESS Sectors 24, 26, 51, 52, 53, which were reduced with the default SPOC pipeline \citep{2020RNAAS...4..201C}, and lightly post-processed with \texttt{lightkurve} \citep{geert_barentsen_2019_2565212}.  These TESS data exhibit an RMS scatter of better than 1 part-per-thousand (ppt) at native 2-minute sampling.  We fit a Keplerian orbit model to the TESS lightcurve using the \texttt{exoplanet} framework \citep{exoplanet:joss}.  We obtained a best fit orbital period of $4.81011$ days, non-zero eccentricity of 0.19, an impact parameter of 0.46, planet-to-star radius of 0.0823, transit depth of 0.74\%, and $T_0$ of 1958.07933 in BTJD.  We did not attempt MCMC sampling to quantify degeneracy among these parameters, and we suspect that the transit is adequately modeled by a circular orbit.  Figure \ref{fig:transit} shows the best-fit orbit overlaid on the detrended TESS lightcurve.  Table \ref{tabOrbit} lists a previous orbit determination and solutions reported here.


\begin{figure}
    \includegraphics[width=\linewidth]{figures/best_fit_orbit.png}
    \caption{Best fit orbit overlaid on the composite TESS lightcurve.  The 1 ppt TESS precision improves upon the ground-based detection and refining the orbital properties and ephemerides.}
    \label{fig:transit}
\end{figure}


\subsubsection{Stellar rotation rate from periodogram analysis}

The TESS Sector 26 lightcurve exhibits a weak $\sim3\;$ppt peak-to-valley out-of-transit modulation on timescales comparable to the exoplanet orbit. TESS Sectors 51-53 do not show as conspicuous a modulation signal as does Sector 26.  Figure \ref{fig:TESSmodulation} shows this modulation in the minimally processed TESS lightcurve.  During the exoplanet orbit fitting, we fit the TESS lightcurve modulation with a quasiperiodic Gaussian Process (GP) model using \citep{celerite} \citep{celerite1,celerite2}.  The GP model fit to the entire composite lightcurve yields a period of 4.68 days; when fit to individual sectors alone the periods hover around 5.9 days.

If we assume the TESS modulation signal is attributable to stellar activity, we obtain a rotation period $P_\mathrm{rot}$ between 4.7 and 5.9 days.  The reported $v\sin{i}=35.8\pm1.1$ km/s would yield an $R_\star \sin{i}$ =


\begin{figure}
    \includegraphics[width=\linewidth]{figures/TESS_S26_trend.png}
    \caption{Stellar lightcurve modulation observed in TESS Sector 26.  The stellar surface pattern speed appears to have a slightly longer period than the exoplanet orbit, causing the transit to occur earlier in the phase of the modulation upon each orbit.}
    \label{fig:TESSmodulation}
\end{figure}






\begin{deluxetable*}{lrrrh}
    \tablewidth{0pc}
    \tabletypesize{\scriptsize}
    \tablecaption{
        Revised orbital and planetary parameters
        \label{tabOrbit}
    }
    \tablehead{
        \colhead{Parameter}   &
        \colhead{Zhou et al. 2017} &
        \colhead{Ivshina \& Winn 2022} &
        \colhead{This Work} &
        \nocolhead{Alternate}
    }
    \startdata
    %%
    ~~~$P$ (days)             \dotfill    & $4.8101025_{-3.3\times 10^{-7}}^{+4.3\times 10^{-7}}$ & 4.8101046$\pm1.7\times10^{-6}$ & 4.81011 & z \\
    ~~~$T_c$ (${\rm BTJD}$)  \dotfill    & $-1038.61533_{-0.00064}^{+0.00076}$ & 1958.08059$\pm$0.00032 & 1958.07933 & z \\
    ~~~$T_{14}$ (hours)  \dotfill    & $6.9888 \pm 0.046$ & x & 7.0636 & z \\
    ~~~$T_{12} = T_{34}$ (days)  \dotfill    & $0.0229\pm0.0010$ & x & y & z \\
    ~~~$\arstar$              \dotfill    & $5.691_{-0.124}^{+0.057}$ & x & y & z \\
    ~~~$\rpl/\rstar$          \dotfill    & $0.0834\pm0.0017$ & x & 0.08231 & z \\
    ~~~$b \equiv a \cos i/\rstar$
    \dotfill    & $0.12_{-0.08}^{+0.12}$ & x & 0.45616 & z \\
    ~~~$i$ (deg)              \dotfill    & $88.8_{-1.3}^{+1.1}$ & x & y & z \\
    \enddata
\end{deluxetable*}


\subsection{HPF analysis with \texttt{muler} }

The raw 2D echellograms were reduced with the \texttt{Goldilocks} package\footnote{\url{https://github.com/grzeimann/Goldilocks_Documentation}}, which outputs 1D extracted spectra for the target, and two reference fibers: blank sky and a laser frequency comb (LFC).  The observations were acquired with the LFC turned off, so this unilluminated spectrum was discarded.

The sky fiber and target fiber have slightly different throughputs and illumination properties so the sky fiber must be scaled prior to subtraction from the target spectrum.  On average the target fiber sees $93\%$ of the flux of the sky reference fiber, and in detail this ratio depends on wavelength.  We quantified the wavelength dependence by acquiring calibration observations of blank sky in both the target and sky fiber.  We applied this wavelength-based scale factor to each target spectrum's associated reference sky fiber to achieve near-imperceptible sky-line residuals.  A few lines exhibit residuals which may arise from slight differences in the local atmospheric conditions between the target and sky fiber.

We masked spectral regions predicted to have significant telluric lines with a template generated by \texttt{telfit} \citep{2014AJ....148...53G}.  We shifted the spectral coordinates to their common barycentric-corrected reference frame \citep{2014PASP..126..838W} as implemented in \texttt{astropy} \citep{2013A&A...558A..33A,2018AJ....156..123A}.

We flatten the spectral to continuum indices near the He 10830 features and compute the equivalent width (EW) defined over the spectral region \censorbox{xx,xxx}$-$\censorbox{xx,xxx} shown in Figure \censorbox{XX}.

The sky-subtraction, telluric masking, barycentric correction, and all of our other standard pre-processing steps are implemented in a new open-source Python interface \texttt{muler}\footnote{\url{https://github.com/OttoStruve/muler/}} (Gully-Santiago \emph{et al.}, submitted).

\authorcomment1{Add overplots of He and Ca II.}

\subsection{EW Uncertainty quantification with MCMC}
We assessed the uncertainty in our equivalent fit Gaussian line profiles to the He 10830 line with \texttt{emcee} \citep{foreman13}. The model consists of a straight-line trend with a Gaussian absorption line.  It has five parameters: the straight-line trend slope $m$ and offset $b$, and the Gaussian amplitude $A$, center wavelength $\mu$, and width $w$.

We computed the equivalent width with the closed-form integral:

\begin{gather}
    W_\lambda=\frac{Aw}{m \mu + b}\cdot\sqrt{2\pi}
\end{gather}

Ultimately the MCMC method was highly accurate but imprecise, and was replaced in favor of \emph{data-driven} methods.



Table \ref{tab:sample_properties} lists the bulk physical properties for HAT-P-67.

\begin{deluxetable*}{ccccc}[!ht]
    \tablecaption{Stellar properties \label{tab:sample_properties}}
    \tablehead{
        \colhead{$T_{\mathrm{eff}}$ (K)} & \colhead{$\log{g}$} & \colhead{RV (km/s)} & \colhead{V sin(i) (km/s)} & \colhead{Reference}}
    \startdata
    $6406^{+65}_{-61}$ & $3.854^{+0.014}_{-0.023}$ & -1.4 $\pm$ 0.5 & 35.8 $\pm$ 1.1 & \cite{2017AJ....153..211Z}\\
    \enddata
\end{deluxetable*}


\begin{figure}
    \includegraphics[width=\linewidth]{figures/stellar_periodogram.png}
    \caption{Periodogram of solely the stellar variation of the flux. }
    \label{fig:stellar_periodogram}
\end{figure}


\subsection{HPF MCMC}
The methodology used to fit the HPF spectra was the Markov chain Monte Carlo (MCMC). After deblazing and normalizing the spectrum, we used a Gaussian line to fit the spectra with five parameters: width, center wavelength, amplitude, slope, and offset.  We focused on each sub-region of the spectrum by cutting of data outside of the $99.9\%-100.1\%*line$ threshold to create a better fit for the data. This threshold was determined as a result of which line we were looking at. The generative model consists of a straight-line trend with a Gaussian subtracted from it. The continuum of the spectrum was given by $continuum=m(\lambda-\lambda_{int})+b$ and the Gaussian fit was subtracted off of the continuum to get the absorption line fit. After applying values to the five parameters for the generative model, we improved our fitting with the MCMC process. We computed the log likelihood of the guessing parameters to determine a quality of fit metric. From there, we ran MCMC with the package \textit{emcee} at 5000 steps for the five parameters. After thinning out and discarding the last 1000 steps, we gathered the final draws for the given parameters with their corresponding uncertainties.

\subsection{HPF Equivalent Width Determination}
In addition to the MCMC method, we verified the precision of our numbers through another fitting method. While the MCMC method produced an accurate result, it lacked the precision required. Thus, we chose to analyze the data again with the summation method.

After collecting the new parameters, we then calculated the equivalent width of the corresponding line with the formula $EW=\sqrt{2\pi}\frac{Aw}{m(\mu-line)+b}$. We then calculated the mean and standard deviation of the equivalent width.
\subsection{Contemporaneous Stellar Activity Quantification}

\begin{deluxetable*}{lcccccc}[!ht]
    \tablecaption{Equivalent Width \label{tab:ew_properties}}
    \tablehead{
        \colhead{Star Name} & \colhead{Date} & \colhead{ID} & \colhead{Observed Line Center Position} & \colhead{Intrinsic Line Center Position} & \colhead{Equivalent Width (\AA)}}
    \startdata
    HAT-P-67 & Aug 7, 2020 & & $10328.493\pm0.017$ & & $0.1607\pm0.0059$\\
    HAT-P-67 & Aug 8, 2020 & & $10345.007\pm0.019$ & & $0.1122\pm0.0068$\\
    HAT-P-67 & Aug 8, 2020 & & $10344.993^{+0.015}_{-0.016}$ & & $0.1064\pm0.0051$\\
    \enddata
\end{deluxetable*}

\section{Results}
\subsection{Detection of Helium with Orbital Phase}
\begin{figure}
    \includegraphics[width=\linewidth]{figures/TESS_EW_HAT-P-67.jpg}
    \caption{A boat.}
    \label{fig:boat1}
\end{figure}
We can see that there is indeed a helium signature that arises when the planet is in transit with its host star.
\subsection{Stellar Activity (non)detection versus Equivalent Width}
There is a likelihood that a portion of the exoplanet emission spectra originates from stellar activity. Thus, we must identify which stellar features mimic a helium signature. Such features can be narrowed down to the star's radial velocity, star spots, and faculae.
\subsection{Star Planet Interaction}
Characterizing a host star's stellar environment is of utmost importance if we want to understand how exoplanet atmospheres are affected. While cool dwarfs generate high irradiation in the ultraviolet and stellar winds, the sub-giant category
\subsection{Gas Motion Relative to Planet}
\subsection{Parker Winds Model}
\subsubsection{p-winds model failure}
\subsection{Geometry of the Escaping Exosphere}
We can depict the transit of HAT-P-67b from the orbital phase versus equivalent width plot. The plot exhibits helium absorption occurring before the planet is in-transit. We can attribute this shallow increase in the equivalent width to the shape of the helium exosphere. Because the absorption gradually increases, we created a leading tail that begins transit 2 days before the planet does. As a result, we mold the shape of the helium exosphere to mimic the irregular absorption with constraints on the velocity and length

\section{Discussion}
\subsection{Hot Spot Scenario}
Using the contemporaneous TESS lightcurve, we notice that there is a 0.2\% flux variation. Assuming that this increase in flux is due to a hot spot with a contrast of 200\%, the hot spot must have a radius that is 12.49 times that of the radius of the Earth. Assuming that the mass loss rate of the planet is its mass over 10 billion years, we can assume that this hot spot is not caused by in falling mass from the planetary envelope as the mass is not enough to cause a 0.2\% modulation in the flux.

\subsection{Tidal Locking Calculation}
From section 3.3.1, we determined the rotational period of the star from contemporaneous TESS data. Given the orbital period of the planet, we concluded that the system is tidally locked.

\subsection{Mass Loss Rate}
Parker wind models, such as those implemented in the Python tool p-winds, are incapable of accurately measuring mass loss rates for systems as extreme as HAT-P-67. Instead, we must constrain the mass loss rate to the mass of the planet over the course of the age of the universe. The resulting value is $2.045*10^{12}$ g/s.

The first scenario is where the mass from the exosphere falls onto the surface of the star. The TESS lightkurve for this system exhibits a reoccurring 0.2\% increase in the flux that is not attributable to the planet’s transit. Assuming that this modulation is due to a hot spot on the stellar surface, we conclude that gravitational potential energy alone is not sufficient to increase the flux by such an amount. Larger mass loss rates could cause brighter hotspots, but this would cause the planet to disappear entirely over its lifetime, which is unphysical.

\subsection{Confinement to Stellar Wind}
\subsection{Limits of Stellar Activity Metrics}
\authorcomment1{3D Heat Map Interpolation figure that mimics MacLeod paper}
\subsection{Comparison to Other Systems}
\authorcomment1{Parameter plot with HAT-P-67 system with others from exoplanet archive}
\section{Conclusions}

Reiteration here.

\clearpage
\pagebreak


\appendix
%\section{HPF spectra post-processing}
%\label{methods-details}


\begin{acknowledgements}
    This research has made use of NASA's Astrophysics Data System.
\end{acknowledgements}

\facilities{HET (HPF), TESS, ASAS}

\software{  \texttt{pandas} \citep{mckinney10},
    \texttt{emcee} \citep{foreman13},
    \texttt{matplotlib} \citep{hunter07},
    \texttt{numpy} \citep{2020NumPy-Array},
    \texttt{scipy} \citep{2020SciPy-NMeth},
    \texttt{ipython} \citep{perez07},
    \texttt{seaborn} \citep{waskom14},
    \texttt{astropy} \citep{2013A&A...558A..33A,2018AJ....156..123A},
    \texttt{muler} (Gully-Santiago et al., submitted),
    \texttt{telfit} \citep{2014AJ....148...53G},
    \texttt{exoplanet} \citep{exoplanet:joss},
    \texttt{jupyter} \citep{Kluyver2016jupyter},
}


\clearpage


\bibliography{ms}

\end{document}
